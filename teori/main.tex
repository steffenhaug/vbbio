\documentclass[12pt]{article}
% Document Setup
% ==============

\author{Steffen Haug}
%% remove the temporary files (.bcf, .aux, ...) if
%% you change the language, because this invalidates
%% the cached babel settings.
\usepackage[nynorsk]{babel}

\usepackage{csquotes}
\usepackage[style=phys]{biblatex}
\usepackage{scalerel}
\usepackage{relsize}

\usepackage{mathtools} % coloneqq
\usepackage{siunitx}
\usepackage{textcomp}

\usepackage{unicode-math}

% Get mathbb and mathcal back (unicode-math replaces them)
\DeclareMathAlphabet{\mathcal}{OMS}{cmsy}{m}{n}
\let\mathbb\relax % remove the definition by unicode-math
\DeclareMathAlphabet{\mathbb}{U}{msb}{m}{n}

\setmainfont{EB Garamond}
\setmonofont{Courier New}
\setmathfont[StylisticSet={7, 2}]{Garamond-Math}


% Custom Commands
% ---------------
\let\vec\mathbf
\newcommand{\deldel}[2]{\frac{\partial #1}{\partial #2}}
\newcommand{\deldelN}[3]{\frac{\partial^{#3} #1}{\partial #2 ^{#3}}}
\newcommand{\dd}[2]{\frac{\mathrm d #1}{\mathrm d #2}}
\newcommand{\textdd}[2]{{\mathrm d #1}/{\mathrm d #2}}
\newcommand{\Tay}[3]{T_{#2}\left \{ #1 \right \}(#3)}
\newcommand{\LRSeq}[1]{\left\{ #1 \right\}}
\newcommand{\Seq}[1]{\big\{ #1 \big\}}

%% fourier transform
\newcommand{\FT}[2]{
    \frac{1}{\sqrt{2\pi}} \smashoperator{\int\limits_{#1=-\infty\mathstrut}^{#1=\infty\mathstrut}}
        #2 e^{-i\omega #1} \D #1
}

\newcommand{\Err}[1]{\mathrm{error}_{#1}}
\newcommand{\Diam}{\mathrm{diam}}
\newcommand{\D}{\mathrm{d}}

\newcommand{\R}{\mathbb{R}}
\newcommand{\N}{\mathbb{N}}
\newcommand{\C}{\mathbb{C}}

\newcommand{\Norm}[2][]{\left\lVert#2\right\rVert_{#1}}
\newcommand{\Jac}[1][]{\mathbf{J}_{#1}}

\newcommand{\Fig}[1]{\mbox{\scshape figure \ref{fig:#1}}}
\newcommand{\Tab}[1]{\mbox{\scshape table \ref{table:#1}}}

\DeclareMathOperator*{\Sub}{\Bigg\vert}

\newcommand{\satan}{\rotatebox[origin=c]{180}{$\dagger$}}

% -- Text Formatting
\linespread{1.10}

% -- Custom Titles
\usepackage{titlesec}

\renewcommand\thesection{\Roman{section}}
\renewcommand\thesubsection{\normalfont\roman{subsection}}
\titleformat{name=\section,numberless}[block]{\Large\scshape}{}{0pt}{}
\titleformat{name=\section}[block]{\Large\scshape}{\thesection.}{5pt}{}

\titleformat{name=\subsection,numberless}[block]{\filcenter\large\scshape}{}{0pt}{}
\titleformat{name=\subsection}[block]{\filcenter\large\scshape}{\thesubsection.}{5pt}{}

\usepackage{moresize}
\usepackage{abstract}
\usepackage{appendix}

\renewcommand{\abstractnamefont}{\normalfont\scshape}
\renewcommand{\abstracttextfont}{\normalfont\small}

% -- Title Section
\usepackage{titling} % Customising the title section
\setlength{\droptitle}{-5\baselineskip} % Move the title up

% Document margins optimized for two-column layout; they
\usepackage[
    marginparwidth=35mm,
    right=15mm,
    left=15mm,
    head=15pt,
]{geometry}
\setlength{\columnsep}{5mm}
\usepackage{afterpage}
\usepackage{pagecolor}

% -- Headers and footers
\usepackage{fancyhdr}
\pagestyle{fancy}
\fancyhead{}
\fancyfoot{}
\fancyfoot[C]{\thepage}


% clean up the "plain" style.
\fancypagestyle{plain}{\
  \fancyhf{}
  \renewcommand{\headrulewidth}{0pt}
}


\usepackage{float}
\usepackage{multicol}
\usepackage{graphics}

% -- Custom captions
\usepackage[hang,small,labelfont=bf,up,textfont=it,up]{caption}

% -- Compact lists
\usepackage{enumitem}
\setlist[itemize]{noitemsep}

% -- LISTINGS --
\usepackage{listings}
% Options for all coding environments
\lstset{
    basicstyle=\scriptsize\ttfamily
}
\lstset{
    frame=tb,
    framexleftmargin=2.5em,
    numbers=left,
    xleftmargin=2.5em,
    showstringspaces=false,
}

% The built-in python highlighter is for python2,
% so we need to modify it:
\newcommand\pythonstyle{\lstset{
language=Python,
morekeywords={self, yield, as, assert, pass},
}}

% python env.
\lstnewenvironment{python}[1][]
{
\pythonstyle
\lstset{#1}
}
{}

% inline pytohn env.
\newcommand\pythoninline[1]{{\pythonstyle\lstinline!#1!}}

\usepackage{xcolor}
\usepackage{tikz}
\usepackage[mode=buildnew]{standalone}

\usepackage{caption}
\captionsetup{format = plain, font = footnotesize, labelfont = {sc}}

\usepackage{multicol}
\usepackage{marginnote}



\title{Initialverdi-oppgåver}


\begin{document}
\maketitle
\begin{multicols*}{2}\noindent % somehow the first line is indented...
    \section{Diffusjonslikninga}
    Gitt initialverdiproblemet
    \begin{equation}
        \phi(x, 0) = \delta(x - x_0)
        \label{phi0}
    \end{equation}
    \begin{equation}
        \deldel{\phi}{t} = D\deldelN{\phi}{x}{2}
        \label{diffusion}
    \end{equation}
    vil vi finne $\phi(x, t)$.
    Eg går ut frå at $x_0 = 0$.
    Dette går bra: So lenge ingen potensial er til stades
    vil det same skje med ei fordeling om $x = x_0$ som med den
    same fordelinga om $x = 0$, så vi kan berre flytte
    koordinatane i etterkant.

    Vi treng hugse nokre få ting.
    Fourier-framstilling av $\delta$-funksjonen:
    \begin{equation}
        \delta(x) = \frac{1}{2\pi}
        \smashoperator{\int\limits_{k = -\infty}^{k = \infty}}
        \D k e^{-ikx}
        \label{deltafn}
    \end{equation}
    Taylor-utvikling av $\phi$ om $t = 0$:
    \begin{equation}
        \phi(x, t) = \sum_{n = 0}^\infty
        \frac{t^n}{n!} \cdot \smashoperator{\Sub\limits_{t = 0}} \deldelN{\phi}{t}{n}
        \label{tay}
    \end{equation}
    Leibniz' integral-regel:
    \begin{equation}
        \DD{x} \left[\;
            \smashoperator[r]{\int\limits_{k=a}^{k=b}} \D{k} f(x, k)
        \right]
        =
        \smashoperator[r]{\int\limits_{k=a}^{k=b}} \D{k} \deldel{f}{x}
        \label{papa}
    \end{equation}
    (bytte om rekkefølga på derivasjon og integrasjon,
     so lenge integrasjonsgrensene er uavhengige av
     variablen vi deriverer med omsyn til)
     Frå \eqref{diffusion} og \eqref{phi0} har vi:
    \begin{equation}
        \deldelN{\phi}{t}{n} = D^n \deldelN{\phi}{x}{2n}
        \;\text;\quad
        \smashoperator{\Sub\limits_{t=0}} \deldelN{\phi}{t}{n}
        = D^n \deldelN{}{x}{2n} \delta(x)
        \label{diffusionN}
    \end{equation}
    som er det siste vi kjem til å få bruk for.

    \subsection*{Integral-framstilling}
    Skriv om \eqref{tay} v. h. a. \eqref{diffusionN}:
    \[
        \phi(x, t) = \sum_{n = 0}^\infty
        \frac{t^n}{n!} \left[ D^n \deldelN{}{x}{2n} \delta(x) \right]
    \]
    Utvid $\delta(x)$ v. h. a. \eqref{deltafn},
    og bytt integrasjon- og derivasjonsrekkefølgje. \eqref{papa}
    Eg droppar å skrive grensene; dei endrar seg ikkje.
    \begin{align*}
        \phi(x, t) &= \sum \left[
        \frac{t^n}{n!} \frac{1}{2\pi} D^n \int
    \D{k} \deldelN{}{x}{2n} e^{-ikx} \right] \\
                   &= \frac{1}{2\pi} \sum \left[ \int \D{k} \frac{t^n}{n!} D^n (-ik)^{2n} e^{-ikx} \right] \\
                   &= \frac{1}{2\pi} \int \D{k} e^{-ikx} \sum \left[ \frac{\left(-Dtk^2\right)^n}{n!} \right]\\
                   &= \frac{1}{2\pi} \int \D{k} e^{-ikx} e^{-Dtk^2}
    \end{align*}
    Skriv på grensene, og gjer koordinatskiftet:
    \begin{equation}
        \phi(x, t) = \frac{1}{2\pi}
        \smashoperator{\int\limits_{k = -\infty}^{k = \infty}}
        \D{k} e^{-ik(x - x_0)} e^{-Dtk^2}
        \label{phi_int}
    \end{equation}

    \subsection*{Normalfordeling}
    Vil evaluere integralet \eqref{phi_int}.
    \begin{align}
    \begin{split}
        \phi(x, t) &= \frac{1}{2\pi}
        \smashoperator{\int\limits_{k = -\infty}^{k = \infty}}
        \D{k} e^{-ik(x - x_0)} e^{-Dtk^2} \\
                   &= \frac{1}{2\pi}
        \smashoperator{\int\limits_{k = -\infty}^{k = \infty}}
        \D{k} e^{-(ak^2 + 2bk)} \:\text;\:
        \begin{array}{l} a = Dt \\  b = \frac{i}{2}(x - x_0) \end{array} \\
                   &\stackrel{\satan}{=} \frac{1}{2\pi} \sqrt{\frac{\pi}{a}} \exp \left(\frac{b^2}{a} \right) \\
                   &= \frac{1}{\sqrt{4\pi D t}} \exp \left( -\frac{(x - x_0)^2}{4 D t} \right)
    \end{split}\label{gauss}
    \end{align}
    Som er det vi skulle vise. (\satan) er frå Rottmann.

    \subsection*{Fysisk tolkning av diffusjonskoeffisienten}
    Tolkar vi \eqref{gauss} som ein sannsynstettleik, er
    variansen til tettleiken $\sigma^2 = 2Dt$.
    Fordelinga vert med andre ord {\em flatare} etter kvart
    som tida går framover, og jo høgare $D$ er, jo raskare
    skjer utflatinga.
    $D$ må derfor tolkast som {\em ``farten''} til diffusjonsprosessen.

    \subsection*{Tidsutvikling av vilkårleg initialverdi}
    Vil løyse initialverdiproblemet
    \begin{equation}
        \phi(x, 0) = g(x)
        \label{generisk_phi0}
    \end{equation}
    og \eqref{diffusion}. Med andre ord betraktar vi vilkårlege
    initialverdiar.
    Eg ser ikkje korleis vi kan bruke hintet på en ``enkel og grei måte'',
    så eg brukar ein betre måte. (:-)) Ta Fourier-transformasjonen av \eqref{diffusion}.
    \begin{equation}
        \label{phihat}
        \hat\phi = \mathscr F\left\{\phi\right\} = \FT{x}{\phi}
    \end{equation}
    Ved hjelp av \eqref{papa}:
    \begin{equation}
        \label{ddtphihat}
        \begin{split}
            \mathscr F \left\{\deldel{\phi}{t}\right\}
                &= \FT{x}{\deldel{\phi}{t}} \\
                &= \deldel{}{t} \left[ \FT{x}{\phi} \right] \\
                &= \deldel{\hat\phi}{t}
        \end{split}
    \end{equation}
    Frå tabell:
    \begin{equation}
        \label{ddxxphihat}
        \mathscr F \left\{D\deldelN{\phi}{x}{2}\right\} = -D \omega^2 \hat \phi
    \end{equation}
    Ved å kombinere \eqref{phihat}, \eqref{ddtphihat} og \eqref{ddxxphihat}
    har vi fått ei ordinær differesiallikning,
    \[
        \deldel{\hat\phi}{t} = -D \omega^2 \hat \phi \text,
    \]
    som vi kjenner løysinga til, gitt initialverdibetingelsen \eqref{generisk_phi0}:
    \[
        \hat \phi(\omega, t) = \hat g(\omega) e^{-Dt\omega^2} \text.
    \]
    Høgresida er eit produkt av to Fourier-transformerte funksjonar,
    så $\phi$ er ein konvolusjon
    \begin{equation}
        \label{phi_conv}
        \phi(x, t) = \frac{g * f}{\sqrt{2\pi}} \;;\quad
        f = \mathscr F^{-1}\left\{e^{-Dt\omega^2}\right\}
             \text.
    \end{equation}
    Invers Fourier-transformasjon:
    \begin{equation}
        \label{Finv_of_exp}
        \mathscr F^{-1}\left\{e^{-Dt\omega^2}\right\}
        = \frac{1}{\sqrt{2Dt}} \exp \left( -\frac{x^2}{4Dt} \right)
    \end{equation}
    Slår saman \eqref{phi_conv} og \eqref{Finv_of_exp}:
    \begin{equation}
        \begin{split}
            \phi(x, t) &= \frac{1}{\sqrt{2\pi} \sqrt{2 D t}}
            \left[
                g(x) * \exp \left( -\frac{x^2}{4Dt} \right)
            \right] \\
                       &= \frac{1}{\sqrt{4 \pi D t}}
                       \smashoperator{\int\limits_{z = -\infty\mathstrut}^{z = \infty\mathstrut}}
                       \D{z} g(z)
                       \exp \left( -\frac{(x - z)^2}{4Dt} \right)
        \end{split}
    \end{equation}
    Dette er en superposisjon av skalerte bell-kurver langs $x$-aksen.

\end{multicols*}
\end{document}
